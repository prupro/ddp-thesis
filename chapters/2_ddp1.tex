
We use stochastic geometry to analyze the performance of a partial decode-and-forward (PDF) relaying scheme applied in a user-assisted relaying setting, where an active user relays data through another idle user in uplink cellular communication. We present the geometric model of a network deploying user-assisted relaying and propose two geometric cooperation policies for fast and slow fading channels. We analytically derive the cooperation probability for both policies. This cooperation
probability is further used in the analytical derivation of the
moments of inter-cell interference power caused by system-wide
deployment of this user-assisted PDF relaying. We then model the
inter-cell interference power statistics using the Gamma distribution by matching the first two moments analytically derived. This
cooperation and interference analysis provides the theoretical
basis for quantitatively evaluating the performance impact of
user-assisted relaying in cellular networks. We then numerically
evaluate the average transmission rate performance and show
that user-assisted relaying can significantly improve per-user
transmission rate despite the increased inter-cell interference. This
transmission rate gain is significant for active users near the
cell edge and further increases with higher idle user density,
supporting user-assisted relaying as a viable solution to crowded
population areas.

\section{Introduction}

\subsection{Motivation}
Mobile subscribers operators and continual driven customer by the increasing demand for number new and
 of
better services place pressing requirements on the underlying
wireless technologies to provide high data rates and wide
coverage. Future generation networks that promise higher data
rates and multifold increase in system capacity include 3GPP
Long Term Evolution-Advanced (LTE-A, 4G) and the emerging 5G systems. The fourth generation (4G) wireless systems were designed to fulfill the requirements of the International Mobile Telecommunications - Advanced (IMT-A). LTE as
a practical 4G wireless system has been recently deployed
in some countries and LTE-A is expected to be deployed
soon around the globe. It is well established that 4G
networks have just reached the theoretical limit on the data
rate with current technologies. These technologies are being
complemented in the fifth generation (5G) wireless systems
by designing and developing new radio concepts to accommodate higher data rates, larger network capacity, higher energy efficiency, and higher mobility necessary to meet the new and challenging requirements of new wireless applications.
5G wireless systems are expected to support peak data rate
of 10 Gb/s for low mobility and 1 Gb/s for high mobility.
These networks are expected to be standardized and deployed around and beyond 2020. Various promising technologies are
proposed for 5G wireless communication systems such as
massive MIMO, energy-efficient communications, Device-to-
Device (D2D) communications, millimeter-wave (mmWave),
and cognitive radio networks.
\par 
D2D and Relaying cooperative communications will play
important roles in future generations wireless networks. D2D
communications enable two proximity users to transmit signal
directly without going through the base station; subsequently,
5G wireless systems are expected to relax the restrictions on
the need to route all user data through the core network. D2D
communications can increase network spectrum utilization and
energy efficiency, reduce transmission delay, offload traffic for
the base station, and alleviate congestion in the cellular core
networks, which make it a promising technology for future
wireless systems. Relay-aided cooperative communi-
cation techniques represent another promising technology that
improves performance in poor coverage areas by enabling
ubiquitous coverage even for users in the most unfavorable
channel conditions. The latest release of the LTE standard
allows the deployment of fixed wireless relays to help cell-
edge mobiles. Yet, other advanced cellular relaying modes are
expected in 5G systems to improve the topology and robustness
of a cellular network and decrease power consumption. These
new technologies include mobile relaying, multi-hop relaying,
and user-equipment based (user-assisted) relaying enabled by
D2D communications 

\subsection{Organization of this report}
This report is in further divided into 6 sections. The topics are organized as follows.
\par Section 2 will discuss partial decode and forward relaying in a standalone setup. The two phases of transmission, signal design and achievable rate of the scheme are presented.
 \par Section 3 is about deployment of the relaying scheme discussed in section 2 in a cellular network. Inter-cell interference and equivalent standard channel model are discussed.
 \par Section 4 contains policies to determine whether the nearest idle neighbour to the active UE can be picked as a relay. Two policies were proposed and analytic expressions for cooperation probabilities were derived.
 \par Section 5 Analytic expressions for interference powers were developed and shape and scale parameters of Gamma distribution that fits interference power statistics were calculated.
 \par Section 6 discusses the simulation setting and how it is different from the theoritical method and the values of parameter used in simulation were listed. It also has simulation results with brief explanations of the result.
 \par Section 7 lists three different topics I'm planning to work on during the second phase. \\
All links/references to equation or figures in this report are clickable.

\pagenumbering{arabic}
\newpage
\section{Partial Decode-and-Forward Relaying}
In this section, we discuss the signal design, channel model and achievable rate of PDF relaying scheme.
\subsection{Signal Design}
Consider a source $\mathcal{S}$, its relay $\mathcal{R}$ and the destination $\mathcal{D}$. Each transmission block is divided into two phases: 1. broadcast transmission in which $\mathcal{S}$ broadcasts to both $\mathcal{R}$ and $\mathcal{D}$. 2. multiple access transmission in which both $\mathcal{S}$ and $\mathcal{R}$ transmit to $\mathcal{D}$. In each block of transmission, $\mathcal{S}$ splits its information into a common part and a private part. The common part is encoded via $U_s^b$ in the 1st phase and $U_s^{m_1}$ in the 2nd phase; and the private part is encoded via $V_s^{m_2}$ in the 2nd phase. The relay $\mathcal{R}$ decodes the information sent by $\mathcal{S}$ in first phase and encodes the same information using $U_s^{m_1}$ in the 2nd phase. \\ 
\begin{figure}[H]
\begin{center}
\includegraphics[height = 2in,width=5in,angle=00]{images/pdfRelaying.png}
\caption{\small Transmission phases in PDF relaying}
\label{fig:sysModel}
\end{center}
\end{figure}
The signals transmitted by $\mathcal{R}$ and $\mathcal{S}$ are as follows:
\begin{align}
\text{Phase 1:}\quad x^b_s &= \sqrt{P_s^b} U_s^b, \label{eq:tranSig1}\\
\text{Phase 2:}\quad x_r^m &= \sqrt{P_r^m}U_s^{m_1}, \label{eq:tranSig2}\\ 
 x^m_s &= \sqrt{P_s^{m_1}}U_s^{m_1} + \sqrt{P_s^{m_2}}V_s^{m_2} \label{eq:tranSig3}
\end{align}
All codewords above are picked from independent Gaussian codebooks with zero mean and unit variance. \\ \\
\textbf{Power Constraints:} Let $P_s$ and $P_r$ be the transmit powers of $\mathcal{S}$ and $\mathcal{R}$ respectively and $\alpha_1$ be the fraction of transmission time allocated to first phase, then the following average power constraints should to be satisfied:
\begin{equation}
\alpha_1 P_s^b + \alpha_2 P_s^m = P_s,\quad \alpha_2P_r^m = P_r
\end{equation}
where $\alpha_2 = 1-\alpha_1$

\subsection{Channel Model}
Considering the transmit signals presented above and assuming flat fading over the two phases, the received signals at $\mathcal{R}$ and $\mathcal{D}$ during first phase are 
\begin{equation}
Y_r^b = h_{sr}x^b_s + Z_r^b , \quad Y_d^b = h_{sd}x^b_s + Z_d^b
\end{equation}
where $b$ denotes broadcast mode, $Z_r^b$ and $Z_d^b$ are \textit{i.i.d} circularly-symmetric complex gaussians with mean 0 and variance $\sigma^2$  - $\mathcal{CN}(0,\sigma^2)$ that represent noises at $\mathcal{R}$ and $\mathcal{D}$. \\
Similarly the received signal at $\mathcal{D}$ during second phase can be modelled as 
\begin{equation}
Y_d^m = h_{sd}x^m_s + h_{rd}x_r^m + Z_d^m
\end{equation}
here $m$ denotes multicast transmission; all others have usual meaning.
The above expression is true only if $\mathcal{D}$ has knowledge about the phase offset between $\mathcal{S}$ and $\mathcal{R}$. This assumption is justified by noting that the phase offset between the two nodes can be estimated at base station.

\subsection{Achievable Rate}
With transmit signals in equations~\ref{eq:tranSig1}-~\ref{eq:tranSig3} and joint ML decoding rule at $\mathcal{D}$, the achievable rate for this relaying scheme is:
\begin{equation} \label{eq:rate}
R_{PDF} \leq min(C_1+C_2,C_3)
\end{equation}
\begin{align}
\text{where } C_1 &= \alpha_1 \log\Big(1+|h_{sr}|^2P_s^b\Big),\\
C_2 &= \alpha_2 \log\Big(1+|h_{sd}|^2P_s^{m_2}\Big),\\
C_3 &= \alpha_1 \log\Big(1+|h_{sd}|^2P_s^b\Big) + \alpha_2\log\bigg(1+|h_{sd}|^2P_s^{m_2} + \Big(|h_{sd}|\sqrt{P_s^{m_1}} + |h_{rd}|\sqrt{P_r^m}\Big)^2\bigg)
\end{align}
$C_1$ represents the rate of the common part that can be decoded at $\mathcal{R}$, $C_2$  the private part that can be decoded at $\mathcal{D}$ provided the common part has been decoded correctly, and $C_3$ both the common and private parts that can be jointly decoded at $\mathcal{D}$. These rates are achievable provided full CSI at all receivers and the source-relay phase offset knowledge.
\par
Now that we know what PDF relaying scheme is and the achievable rate, let us see how this scheme performs in cellular networks. To analyse system performance under PDF relaying, we need to know network geometry i.e., how the users and base stations are distributed, how many users can take advantage of relaying, how users identify a potential relay etc. In the next couple of sections we describe network geometry,  received signals and interference model when relaying is deployed in the whole network, and cooperation policies.

\section{Cellular Network Geometry and User-Assisted Relaying}

\subsection{Network geometry model}
Consider a cellular system which consists of multiple
cells, each cell has a single base station and each base station
serves multiple users. Each of the users uses a distinct frequency
block. Each user is served by the single base station that
is closest to that user.


\par We use stochastic geometry to
describe the uplink cellular network. We
assume that the active users in different cells that use the same resource block and cause interference to each other are distributed on a two-dimensional plane according to a homogeneous and stationary Poisson point process (PPP)
$\Phi_1$ with intensity $\lambda_1$. The set of user equipments(UEs) that are in idle state and can participate in relaying are distributed according to another PPP $\Phi_2$ with intensity $\lambda_2$. We assume $\Phi_1$ and $\Phi_2$ are independent. Furthermore, under the assumption that each BS serves a
single mobile in a given resource block, the BS should be closer to its served UE than to any other UE. Therefore we assume each BS is uniformly distributed in the Voronoi cell of its served UE. Fig.~\ref{fig:netLayout} shows an example layout of the network.

\begin{figure}[H]
\begin{center}
\includegraphics[height = 3in,width=4in,angle=00]{images/netLayoutPaper.png}
\caption{\small Sample layout of a cellular network ($\lambda_2 = 2\lambda_1$)}
\label{fig:netLayout}
\end{center}
\end{figure}

\subsection{Channel Model}
In this section, we describe the channel model when PDF relaying is deployed in cellular network. In this case, there will be out-of-cell inteference in addition to noise. The interference is due to frequency reuse in other cells.
\par Consider $i^{th}$ active UE, we model the received signals at the relay and base station in this cell during 1st phase as
\begin{equation*}
Y_{r,i}^b = h^{(i)}_{sr}x_{s,i}^b + I_{r,i}^b + Z_{r,i}^b,
\end{equation*}
\begin{equation}
Y_{d,i}^b = h^{(i)}_{sd}x_{s,i}^b + I_{d,i}^b + Z_{d,i}^b
\end{equation}
where $I_{r,i}^b$ and $I_{d,i}^b$ represent the interference received at the $i^{th}$ relay and destination. 
\par 
In second phase of the transmission, the received signal at the BS can be modelled as 
\begin{equation}
Y_{d,i}^m = h^{(i)}_{sd}x_{s,i}^m + h^{(i)}_{rd}x_{r,i}^m+ I_{d,i}^m + Z_{d,i}^m
\end{equation}

\subsection{Interference}
To model interference, we assume prefect frame synchronization. LTE-Advanced imposes very strict requirements on synchronization anyway. Interference at the relay during first phase and at the destination(BS) during first and second phases can be expressed as
\begin{equation*} 
I_{r,i}^b = \sum_{k \neq i} B_k h^{(k,i)}_{sr} x_{s,k}^b + (1-B_k)h_{sr}^{(k,i)}x_{s,k} ,
\end{equation*}
\begin{equation*}
I_{d,i}^b = \sum_{k \neq i} B_k h_{sd}^{(k,i)} x^b_{s,k} + (1-B_k)h_{sd}^{(k,i)}x_{s,k},
\end{equation*}
\begin{equation} \label{eq:interferences}
I_{d,i}^m = \sum_{k \neq i} B_k \Big(h_{sd}^{(k,i)} x^m_{s,k} + h_{rd}^{(k,i)} x^m_{r,k}\Big) + (1-B_k)h_{sd}^{(k,i)}x_{s,k}
\end{equation}
the summation is over all active users. Here, $h_{sd}^{(k,i)}$ and $h_{rd}^{(k,i)}$ , respectively, are the channel fading from the $k^{th}$ active UE in $\Phi_1$ and the associated relaying UE in $\Phi_2$ to the BS associated with the $i^{th}$ active UE in $\Phi_1$; and $h_{sr}^{(k,i)}$ is the channel fading from the $k^{th}$ active UE in $\Phi_1$ to the relaying UE associated with the $i^{th}$ active UE in $\Phi_1$.
\par $B_k$ in above expressions is a Bernoulli random variable with success probability $\rho$. $B_k = 1$ is used to indicate the $k^{th}$ active UE's decision to exploit the help of another idle UE, a relay, and apply the relaying transmission strategy, and $B_k = 0$ indicates that the $k^{th}$ UE has no relay. In section~\ref{sec:coop}, we derive the cooperation probability $\rho$ for different cooperation policies.
\par
 For a given setting of nodes locations, based on the
interference model in Eq.~\ref{eq:interferences}, we can use the fact
that interference at either the relay or destination is the
sum of an infinite number of signals undergoing independent fading from nodes distributed in the infinite 2-D plane and use the law of large numbers to approximate the interference as a complex Gaussian distribution.
Also, since the transmitted codewords are complex Gaussian with zero mean, mean of interference is zero. To fully characterize interference as a
complex Gaussian distribution, we define their distributions as $ I_{d,i}^b \sim \mathcal{CN} (0,\mathcal{Q}_{d,i}^b), I_{d,i}
^m \sim \mathcal{CN}(0,\mathcal{Q}_{d,i}^m),$ and $I_{r,i}^b \sim \mathcal{CN}
(0,\mathcal{Q}_{r,i})$ with the variances derived later
in Section~\ref{sec:interference}. The power of these interference terms which
correspond to the variance of the Gaussian random variables
are function of node locations and hence vary with different
network realizations.

\subsection{Equivalent Standard Channel Model}
Using the interference model discussed above, we can convert the channel model in case of relaying into the standard form to capture the effects of
interference into the channel fading as
\begin{align*}
\tilde{Y}_{r,i}^b &= \tilde{h}_{sr}^{(i)}x_{s,i}^b + \tilde{Z}_{r,i}^b, \\
\tilde{Y}_{d,i}^b &= \tilde{h}_{sd}^{(i)}x_{s,i}^b + \tilde{Z}_{d,i}^b, \\
\tilde{Y}_{d,i}^m &= \tilde{h}_{sd}^{(i)}x_{s,i}^m + \tilde{h}_{rd}^{(i)}x_{r,i}^m + \tilde{Z}_{d,i}^m
\end{align*}
where the new channel fading terms are defined as

\begin{equation*}
\tilde{h}_{sr}^{(i)} = \frac{h_{sr}^{(i)}}{\sqrt{\mathcal{Q}_{r,i} + \sigma^2}}, \quad \tilde{h}_{sd}^{(b,i)} = \frac{h_{sd}^{(i)}}{\sqrt{\mathcal{Q}_{d,i}^b + \sigma^2}} \quad
\tilde{h}_{sd}^{(m,i)} = \frac{h_{sd}^{(i)}}{\sqrt{\mathcal{Q}_{d,i}^m + \sigma^2}},
\quad \tilde{h}_{rd}^{(i)} = \frac{h_{rd}^{(i)}}{\sqrt{\mathcal{Q}_{d,i}^m + \sigma^2}}
\end{equation*}
and the noise terms are now all $\mathcal{CN}(0,1)$. Using these equivalent 
standard channels, we can compute the transmission rate using Eq.~\ref{eq:rate}

\section{Cooperation Policies and Probability} \label{sec:coop}
In this section, we look at three cooperation policies: an ideal policy $E_1$, a 
pure geometric policy $E_2$ and a hybrid policy $E_3$ that defines whether an active UE should select an inactive UE to use it in PDF relaying. Also, expressions for cooperation probabilities of $E_2$ and $E_3$ are derived.
\subsection{Policies}
\subsubsection{Ideal Policy $E_1$}
The ideal cooperation policy $E_1$ requires the active UE nodes to know instantaneous  SINRs of the relay link($\mathcal{S}-\mathcal{R}$) and the
direct link($\mathcal{S}-\mathcal{D}$). The policy is defined as 
\begin{align*}
E_1 &= \Big\{|\tilde{h}_{(sr)}^{(k)}|^2 \geq |\tilde{h}_{(sd)}^{(k)}|^2\Big\} \\
&\backsimeq \Big\{ \frac{g_{sr}r_2^{-\alpha}}{\mathcal{Q}_{r,k}} \geq \frac{g_{sd}r_1^{-\alpha}}{\mathcal{Q}_{d,k}^b} \Big\}
\end{align*}
where $r_1$ and $r_2$ denote the direct distance
between $\mathcal{S}$ and $\mathcal{D}$ and cooperation distance between $\mathcal{S}$ and its closest idle UE, respectively and $\alpha$ is pathloss exponent. This event $E_1$ identifies whether an idle UE will be associated as a relay for the $k^{th}$ UE and participate in transmission. Noise variance $\sigma^2$ is ignored since interference power dominates. 
\par Since interference at relay and destination during first phase is more or less the same and $g_{sr}, g_{sd}$ are identically distributed, we can safely ignore them and propose a policy that depends only on distances.

\subsubsection{Pure Geometric Policy $E_2$}
This policy is defined as
\begin{equation}
E_2 = \{r_2\leq r_1, D \leq r_1 \}
\end{equation}
where $D$ is the distance between $\mathcal{R}$ and $\mathcal{D}$. In words, if source's(active UE's) nearest idle neighbour is in the intersection region of two circles of radius $r_1$ centered at source and destination, then that idle UE will be chosen to act as a relay.
\par $E_2$ is more practical than policy $E_1$ in the sense that
it does not require full knowledge of both the channel fading
and the interference at the decision making node. Instead, it
only requires the decision making nodes to know the distances
from the active user to the nearest idle user and to the base
station. It represents a practical decision making strategy for
fast fading channels, requiring no knowledge of the channel
fading. 
\subsubsection{Hybrid Policy $E_3$}
This policy is proposed for slow fading channels where small scale fading parameters estimation and their
feedback to the decision making node is feasible. 
\begin{equation}
E_3 = \{g_{sd}r_1^{-\alpha} \leq g_{sr}r_2^{-\alpha}, D \leq r_1 \}
\end{equation}
Note that this cooperation policy is still independent of the
interference as in the pure geometric cooperation policy $E_2$.

\subsection{Cooperation Probabilities}
In this part of the section we derive cooperation probabilities $\rho_2$ and $\rho_3$ for the policies $E_2$ and $E_3$ respectively. For the ideal policy $E_1$, analytic evaluation of the
cooperation probability is rather complicated because of the
inter-dependency between the cooperation decision and consequential interference among different cells. Consider a random BS and its associated active UE. 
The distribution of the distance
$r_1$ between the $i^{th}$ UE and its associated BS can be shown to
be Rayleigh distributed directly from the null probability of a
two dimensional PPP distribution. 
\par  Due to the stationarity of the
PPP, i.e., location of the origin doesn't change the distribution of points, and the independence of $\Phi_2$ from BSs distribution we can assume that
the location of the UE associated with the BS under study
represents the origin point of $\Phi_2$ . Then, each UE
in $\Phi_1$ chooses the closest UE in $\Phi_2$ to assist it in relaying
its message to the serving BS. Hence, similar to source-to-
destination distance, the distribution of the source-to-relay
distance $r_2$ between the $i^{th}$ UE and its associated relaying UE
can be also shown to be Rayleigh distributed from the null probability of a two dimensional PPP. Therefore,
\begin{equation*}
f_{r_1}(r_1) = 2\pi\lambda_1r_1e^{-\lambda_1\pi r_1^2},
\end{equation*}
\begin{equation}
f_{r_2}(r_2) = 2\pi\lambda_2r_2e^{-\lambda_2\pi r_2^2}
\end{equation}

\begin{theorem}{Cooperation Probabilities.}
The probability of
deploying user-assisted relaying for a randomly located active
user within a cell can be evaluated as follows:
\begin{itemize}
\item[i.] For policy $E_2$
\begin{equation}
\rho_2 = \int_{-\pi/2}^{-\pi/3}\frac{2\lambda_2 cos^2\psi_0}{\pi(\lambda_1+4\lambda_2cos^2\psi_0)}d\psi_0 + \int_{\pi/3}^{\pi/2}\frac{2\lambda_2 cos^2\psi_0}{\pi(\lambda_1+4\lambda_2cos^2\psi_0)}d\psi_0 + \frac{\lambda_2}{3(\lambda_1+\lambda_2)}
\end{equation}
\item[ii.] For policy $E_3$
\begin{align*}
\rho_3 &= \int_0^2 f_{\beta}(z)\int_{-\pi/2}^{-cos^{-1}(z/2)}\frac{2\lambda_2 cos^2\psi_0}{\pi(\lambda_1+4\lambda_2cos^2\psi_0)}d\psi_0 dz \\ 
&+ \int_0^2 f_{\beta}(z)\int^{\pi/2}_{cos^{-1}(z/2)}\frac{2\lambda_2 cos^2\psi_0}{\pi(\lambda_1+4\lambda_2cos^2\psi_0)}d\psi_0 dz \\ &+\int_0^2 f_{\beta}(z)\frac{\lambda_2 z^2 cos^{-1}(z/2)}{\pi(\lambda_1+\lambda_2z^2)}dz \\ 
&+ \int_2^{\infty} f_{\beta}(z)\int_{-\pi/2}^{\pi/2}\frac{2\lambda_2 cos^2\psi_0}{\pi(\lambda_1+4\lambda_2cos^2\psi_0)}d\psi_0 dz
\end{align*}
where $\beta = \bigg(\frac{g_{sr}}{g_{sd}}\bigg)^{1/\alpha}$ and $f_{\beta}(z)$ is pdf of $\beta$ which can be shown to be 
\begin{equation}
f_{\beta}(z) = \frac{\alpha z^{\alpha-1}}{(1+z^{\alpha})^2}
\end{equation}
\end{itemize}
\end{theorem}
\begin{proof}
\begin{itemize}
\item[i.] 
\begin{align*}
\rho_2 &= \mathbb{P}\{E_2\} \\
&= \mathbb{P}\{r_2 \leq r_1, r_1^2+r_2^2-2r_1r_2cos\psi_0 \leq r_1^2\} \\
&= \mathbb{P}\{r_2 \leq r_1, r_2 \leq 2 r_1 cos\psi_0 \} \\
\end{align*}
when $|\psi_0|<\pi/3, ~ r_1 < 2r_1cos\psi_0  \Rightarrow \text{ if } r_2 < r_1 \text{, $r_2$ satisfies both inequalities.}$ Accordingly, we define $\mathcal{E}_1$ and $\mathcal{E}_2$ as follows
\begin{align*}
\mathcal{E}_1 &= (2\pi)^2\lambda_1 \lambda_2 \int_0^\infty \int_0^{2r_1 cos\psi_0}r_1r_2e^{-\pi(\lambda_1 r_1^2 + \lambda_2 r_2^2)}dr_2 dr_1 \\
&= \frac{2\lambda_2 cos^2\psi_0}{\pi(\lambda_1+4\lambda_2cos^2\psi_0)} \\
\mathcal{E}_2 &= (2\pi)^2\lambda_1 \lambda_2 \int_0^\infty \int_0^{r_1}r_1r_2e^{-\pi(\lambda_1 r_1^2 + \lambda_2 r_2^2)}dr_2 dr_1 \\
&= \frac{\lambda_2}{2\pi(\lambda_1+\lambda_2)}
\end{align*}
 
\begin{align*}
\text{Now, }\rho_2 &= \int_{-\pi/3}^{\pi/3} \mathcal{E}_2 d\psi_0 + 2\int_{\pi/3}^{\pi/2} \mathcal{E}_1d\psi_0 \\
&=  \frac{\lambda_2}{3(\lambda_1+\lambda_2)} + 2\int_{\pi/3}^{\pi/2} \mathcal{E}_1d\psi_0
\end{align*} 

\item[ii.]
\begin{align}
\rho_3 &= \mathbb{P}\{E_3\} \\
&= \mathbb{P}\{r_2 \leq \bigg(\frac{g_{sr}}{g_{sd}} \bigg)^{1/\alpha}r_1, r_1^2+r_2^2-2r_1r_2cos\psi_0 \leq r_1^2\} \\
&= \mathbb{P}\{r_2 \leq \beta r_1, r_2 \leq 2 r_1 cos\psi_0 \} \\
&= \mathbb{P}\{r_2 \leq 2 r_1 cos\psi_0 \} \qquad \text{ for } \beta > 2 \\
&= \mathbb{P}\{r_2 \leq \beta r_1\} \qquad \text{ for } \beta < 2  \text{ and } |\psi_0| < cos^{-1}(\beta/2) \\
&= \mathbb{P}\{r_2 \leq 2 r_1 cos\psi_0 \} \qquad \text{ for } \beta < 2 \text{ and } cos^{-1}(\beta/2) < |\psi_0| <  \pi/2 \\
\therefore \rho_3&= 2 \int_0^2 f_{\beta}(z)\int^{\pi/2}_{cos^{-1}(z/2)}\mathcal{E}_1 d\psi_0 dz +\int_0^2 f_{\beta}(z)\int_{-cos^{-1}(z/2)}^{cos^{-1}(z/2)}\mathcal{E}_3 d\psi_0dz \\&+ \int_2^{\infty} f_{\beta}(z)\int_{-\pi/2}^{\pi/2}\mathcal{E}_1 d\psi_0 dz \label{eq:corrected}
\end{align}

$\mathcal{E}_1$ is defined in part i. of the proof and  $\mathcal{E}_3 = \frac{\lambda_2 z^2}{2\pi(\lambda_1+\lambda_2z^2)}$ which is nothing but $\mathcal{E}_2$ with $\lambda_2 = \lambda_2z^2$. $f_\beta(z)$, the pdf of $\beta$, can be obtained as follows

\begin{align*}
F_\beta(z) &= \mathbb{P}\bigg\{ \bigg( \frac{x_1}{x_2}\bigg)^{1/\alpha} \leq z \bigg\} = \mathbb{P} \{ x_1\leq z^\alpha x_2\} \\
&= \int_0^\infty \int_0^{z^\alpha x_2} e^{-(x_1+x_2)} dx_1dx_2 \quad \text{ since } g_{sr},g_{sd} \sim Exp(1) \\
&= 1-\frac{1}{1+z^\alpha}, \qquad z \in [0,\infty)
\end{align*}
The pdf $f_\beta(z)$ is then obtained by differentiating $F_\beta(z)$ :
\begin{equation*}
    f_\beta(z) = \frac{dF_\beta(z)}{dz} = \frac{\alpha z^{\alpha-1}}{(1+z^\alpha)^2} \quad z \in [0,\infty)
\end{equation*}
\end{itemize}
\end{proof}
\section{Interference Analysis} \label{sec:interference}
User-assisted relaying actually increases the amount of out-
of-cell interference in the network as some idle users are now
transmitting when relaying information of active users. It is
therefore necessary to understand this out-of-cell interference
power, particularly its distribution, in order to assess the overall
impact of user-assisted relaying on system performance.
\subsection{First Two Moments of Interference Power} Since it is difficult to describe the
exact distribution of out-of-cell interference power, here we
choose to model the interference power to the cell under study
as a Gamma distribution by fitting the first two moments of
the interference power analytically developed using stochastic
geometry of the field of interferers outside that cell.
The expressions for interference power can be developed from Eqs.~\ref{eq:interferences}. 
\begin{align}
\mathcal{Q}_{d,i}^b &= \sum_{k\neq i}B_k \Big |h_{sd}^{(k,i)}\Big|^2P_{s,k}^b + (1-B_k)\Big|h_{sd}^{(k,i)}\Big|^2P_{s,k} \\
\mathcal{Q}_{d,i}^m &= \sum_{k\neq i}\bigg[B_k\bigg(\Big|h_{sd}^{(k,i)}\Big|^2 P_{s,k}^m+\Big|h_{rd}^{(k,i)}\Big|^2 P_{r,k}^m\bigg)\bigg] + (1-B_k) \Big| h_{sd}^{(k,i)}\Big|^2 P_{s,k} \\
\mathcal{Q}_{r,i} &= \sum_{k\neq i}B_k \Big|h_{sr}^{(k,i)}\Big|^2P_{s,k}^b + (1-B_k)\Big|h_{sr}^{(k,i)}\Big|^2P_{s,k} 
\end{align}
\begin{theorem}{Interference Power Statistics} \label{theorem:theorem2}
For network-wide
deployment of user-assisted relaying, the out-of-cell interfer-
ence generated at the destination BS and the relaying UE have
the following statistics:

\begin{itemize}
\item[i.] 
The first two moments, mean and variance, of interference
power at the destination BS during the 1st and 2nd phase,
respectively, are
\begin{equation}
\mathbb{E}[\mathcal{Q}_{d,i}^b] = \frac{2\pi\lambda_1\zeta_1}{\alpha-2}R_c^{2-\alpha}, \qquad \mathbb{E}[\mathcal{Q}_{d,i}^m] = \frac{2\pi\lambda_1\zeta_3}{\alpha-2}R_c^{2-\alpha}
\end{equation}
\begin{equation}
\text{var}[\mathcal{Q}_{d,i}^b] = \frac{\pi\lambda_1\zeta_2}{\alpha-1}R_c^{2(1-\alpha)}, \qquad \text{var}[\mathcal{Q}_{d,i}^m] = \frac{\pi\lambda_1\zeta_4}{\alpha-1}R_c^{2(1-\alpha)}
\end{equation}

\item[ii.]
The first two moments, mean and variance, of interfer-
ence power at the idle UE associated as a relay with the ith
active UE are
\begin{equation}
\mathbb{E}[\mathcal{Q}_{r,i}] = \lambda_1\zeta_1 \int_0^{2\pi}\int_{R_c}^\infty (r^2+D^2-2rDcos\theta)^{\alpha/2}rdrd\theta
\end{equation}
\begin{equation}
\text{var}[\mathcal{Q}_{r,i}] = \lambda_1\zeta_2 \int_0^{2\pi}\int_{R_c}^\infty (r^2+D^2-2rDcos\theta)^{\alpha/2}rdrd\theta
\end{equation}

\begin{align} \label{eq:zeta1}
\text{where} \quad  \zeta_1 &= \rho_1 P_{s,k}^b +(1-\rho_1)P_{s,k} \\
                \zeta_2 &= 2[\rho_1(P_{s,k}^b)^2 + (1-\rho_1)P_{s,k}^2],\\
                \zeta_3 &= \rho_1 (P_{s,k}^m+P_{r,k}^m) +(1-\rho_1)P_{s,k}, \\
                \zeta_4 &= 2[\rho_1(P_{s,k}^m+P_{r,k}^m)^2 + (1-\rho_1)P_{s,k}^2 -\rho_1P_{s,k}^mP_{r,k}^m] \label{eq:zeta4}
\end{align}
\end{itemize}
\end{theorem}
\begin{proof}
\begin{align*}
\mathbb{E}[\mathcal{Q}_{d,i}^b] &= -\frac{\partial\mathcal{L}_{\mathcal{Q}_{d,i}^b}(s)}{\partial s}\bigg\rvert_{s=0}, \\
\text{var}[\mathcal{Q}_{d,i}^b] &= -\frac{\partial^2\mathcal{L}_{\mathcal{Q}_{d,i}^b}(s)}{\partial s^2}\bigg\rvert_{s=0} - \Big(\mathbb{E}\big[\mathcal{Q}_{d,i}^b \big] \Big)^2
\end{align*}
where $\mathcal{L}_{\mathcal{Q}_{d,i}^b}(s)$ is the Laplace transform of $\mathcal{Q}_{d,i}^b$ and $R_c = 1/2\sqrt{\lambda_1}$ is the cell radius. Means and variances of $\mathcal{Q}_{d,i}^m$, $\mathcal{Q}_{r,i}$ can be calculated similarly.
\end{proof}
\par From the above results for interference power statistics, the interference power is directly proportional to both
the active users density, $\lambda_1$ , and the transmission power levels
represented by $\zeta_i , i \in [1:4] $ in Eqs.~\ref{eq:zeta1} - ~\ref{eq:zeta4}

\subsection{Modelling Interference Power Distribution}
A parameterized probability distribution, which includes a
wide variety of curve shapes, is useful in the representation of
data when the underlying model is unknown or difficult to obtain in closed form. A parameterized probability distribution is
usually characterized by its flexibility, generality, and simplicity. Although distributions are not necessarily determined by
their moments, the moments often provide useful information
and are widely used in practice. It is shown that the Gamma
distribution is a good approximation for the interference when
the point under study is closer to the cell center, but fails
to represent the actual interference distribution whenever the
point under study is exactly at the cell edge. We use the same
approach here and match a Gamma distribution to the first
two moments of the interference power terms derived earlier
in Theorem~\ref{theorem:theorem2}.
\subsubsection{Gamma Distribution}
The Gamma distribution is specified by a shape parameter $k$ and a scale parameter $\theta$. The pdf of a Gamma distributed RV $\gamma[k,\theta]$ is defined as 
\begin{equation*}
F_\gamma(q|k,\theta) = \frac{q^{k-1}e^{(-q/\theta)}}{\theta^k\Gamma(k)}
\end{equation*}
where the Gamma function $\Gamma(t)$ is defined as $\Gamma(t) = \int_0^{\infty}x^{t-1}e^{-x}dx$. The mean and variance of $\gamma[k,\theta]$ are $k\theta$ and $k\theta^2$ respectively.
\par
Since we know mean and variance of interference powers, we can estimate the shape and scale parameters by using the formulae:
\begin{equation}
k_i = \frac{(\mathbb{E}[\mathcal{Q}_i])^2}{\text{var}[\mathcal{Q}_i]}, \theta_i =\frac{\text{var}[\mathcal{Q}_i]}{\mathbb{E}[\mathcal{Q}_i]}
\end{equation}

\section{Simulations and Results}

\subsection{Simulation Setting}

All simulations were done on a square region of side length 200m. 
To generate active UEs in the region, the number of UEs is taken as a realization of poisson RV with parameter $\lambda_1$ and these number of UEs were uniformly distributed in the square region. The same is done to generate idle UEs but with parameter $\lambda_2$. I discarded the UEs whose Voronoi region extends to infinity. 
\par In theory, the base station of a UE is uniformly distributed in the Voronoi region of UE but there is no easy practical way to uniformly pick a point from a polygonal area. One method is to triangulate the polygonal Voronoi region, choose a triangle weighted by area, choose a point inthat triangle. This is clearly quite complex to code so I've not implemented this method. The method I followed to generate BSs is - pick a number greater than or equal to the number of active UEs and distribute these number of BSs uniformly in the square region. Now go to each active UE and check if there are any BSs in its Voronoi region. If there are BSs, pick one of them and associate it with the UE and discard other BSs in the Voronoi region. Since BSs are distributed uniformly over the whole region, the result is as good as picking BSs uniformly in the Voronoi regions of UEs which is what we wanted but there is a catch. In the theoritical method, each UE with a finite Voronoi region is guaranteed to have a BS whereas in the way that I'm generating, some UEs might not have a BS even though their Voronoi region is of finite area. Further, the UEs without a BS are not included in rate or cooperation probability analysis which is logical since without an associated BS, the UEs cannot considered active.
\par For all simulations, we assume that UEs are using maximum power to transmit without applying any power control method.\label{sec:powerCon}  The powers used during the two phases of transmission are as follows
\begin{itemize}
\item Source and relays use equal power $\Rightarrow P_{s,i} = P_{r,i} $ 
\item Source use equal power during broadcast and multicast phases $\Rightarrow P_{s,i}^b = P_{s,i}^m$ 
\item $P_{s,i}^{m_1} = \beta_1 P_{s,i}^m$ and $P_{s,i}^{m_2} = (1-\beta_1) P_{s,i}^m$. Where $\beta_1$ is allocated optimally to maximize the transmission rate of the active user. To do this, rate is expressed as a function of $\beta_1$ and minimized negative rate using MATLAB tool \textit{fminrnd}.
\end{itemize}
\subsection{Results}
\begin{figure}[H]
\begin{center}
\includegraphics[height = 4in,width=7in,angle=00]{images/netLayoutSim.png}
\caption{\small Network Layout}
\label{fig:netLayoutSim}
\end{center}
\end{figure}
This a sample network layout generated by using the method discussed in previous subsection. We can see that some of the UEs are well within the range but have no BS. Only UEs with a BS are considered active. A fraction active UEs have relays, these UEs use PDF relaying. Cooperation probability = number of active UEs with relays/total number of active UEs.
\begin{figure}[H]
\begin{center}
\includegraphics[height = 3in,width=4in,angle=00]{images/corrected.png}
\caption{\small Corrected cooperation probability of $E_3$}
\label{fig:correctedE3}
\end{center}
\end{figure}
In the published paper, the analytic result for cooperation probability of $E_3$ has an error. The correction being using $\mathcal{E}_3$ instead of $\mathcal{E}_2$ in eq.~\ref{eq:corrected}. The corrected analytic result matches the simulation result as can be seen in the above figure. 
\begin{figure}[H]
\begin{center}
\includegraphics[height = 3in,width=4in,angle=00]{images/coopP.png}
\caption{\small Cooperation probabilities of $E_2$, $E_3$ versus user density ratio}
\label{fig:cooP}
\end{center}
\end{figure}
From the above graph we can see that cooperation probability of both policies increases with user density ratio($\lambda_2/\lambda_1$) and reach a maximum of 0.5 for very large user density ratio. Also, the analytic and simulations results closely match. 

\begin{figure}[H]
\begin{center}
\includegraphics[height = 2.8in,width=4in,angle=00]{images/rates.png}
\caption{\small Average rate per user; $\lambda = \lambda_2/\lambda_1$}
\label{fig:rates}
\end{center}
\end{figure}
From the above figure, we can clearly see that average rate per user has increased when PDF relaying is deployed over the whole network. The rate also increases as user density ratio is increased but we caan't sure whether the rate keeps increasing with $\lambda$. It might happen that the rate decreases for very large user ratio density due to increase in interference.
