\chapter{Conclusion and Further Work}
\section{Conclusion}
Based on the results obtained we conclude the following:
\begin{itemize}
\item Based on the work of H. Elkotby, we verified that there is throughput gain when user-assisted relaying is deployed in cellular network despite the increase in interference. 
\item We proposed a downlink cooperation policy following the same logical reasoning that was used for uplink policies.
\item We briefly discussed the importance of relay mobility as a decision parameter in relay selection.
\item We gave a general expression for the probability that a node leaves the feasible region that can be used for any mobility model. 
\item The probability of leaving the region is maximum for points on the boundary. The maximum shifts to right side border for points closer to intersection of circles.
\item The expected number of transitions can be better approximated by discretizing the region as states in a Markov Chain.
\end{itemize}

\section{Further Work}
The following can be looked at in the future.
\begin{itemize}
\item	Using the idea presented in chapter 4 to obtain lower and upper bounds on the expected number of steps if not to find the the value itself to a close apprximation.
\item	Through out the work we took $R_0$, the distance between base station and user, to be constant and did not assume the distribution of inital position of the relay. The distributions of $R_0$ and initial position can be incorporated to find the effect of mobility on the whole network.
\item	To include sojourn time as one of the deciding parameters in relay selection.     
\item	To apply the general methodology used in this work for other mobility models. 
\end{itemize}
