\chapter{Moving Relays}
\section{Mobility Model}
\subsection{Random Waypoint}

\section{Leaving in  one transition}
Let us find the probability with which a node at $(r,\theta)$ leaves the region in 
one transition. Consider the figure \ref{fig:rr1}. 
\begin{figure}[h]
    \centering \vspace{-0.1in}
    \includegraphics[width=0.6\textwidth]{images/geo1.pdf}
    \vspace{-20pt} \caption[Effect of the proximal Operator]{\small Effect of the Proximal Operator \footnotemark}
    \label{fig:rr1}
\end{figure}
\footnotetext{Image Source: \url{http://www.stanford.edu/~boyd/papers/pdf/prox_algs.pdf}}
The node starts at $B$ and let $C$ be 
the destination during first transition. $\alpha, r_1$ are chosen according to the mobility 
model described in the previous section. Whether or not $C$ is outside the region depends on
its distance from the circles' centers $O$ and $C$. $OC$ can be found using cosine rule:
$OC^2 = OB^2 + BC^2 - 2 \cdot OB \cdot BC \cdot \cos(\angle CBO)$. $\angle CBO = \pi-\theta+\alpha$ (see figure \ref{fig:ocac}).
\begin{figure}[h]
    \centering \vspace{-0.1in}
    \includegraphics[width=0.6\textwidth]{images/geo2.pdf}
    \vspace{-20pt} \caption[Effect of the proximal Operator]{\small Effect of the Proximal Operator }
    \label{fig:ocac}
\end{figure}

 Therefore, \begin{equation} OC^2 = r^2 + r_1^2 + 2rr_1\cos(\theta-\alpha) \end{equation} To find $AC$, drop a perpendicular from $C$ to the line $OA$ and let the intersection point be $D$ as shown in figure \ref{fig:ocac}. Consider $\bigtriangleup CDA$, 
\begin{align*}
CD &= OB\sin\theta + BC\sin\alpha \\
&= r\sin\theta + r_1\sin\alpha \\
DA &= OA - OB\cos\theta - BC\cos\alpha \\
&= R_0-r\cos\theta-r_1cos\alpha
\end{align*}
Since $\angleCDA = \pi/2$, $AC^2 = CD^2 + DA^2$. $AC$ can therefore be given by 
\begin{equation}
	AC^2 = (r\sin\theta + r_1\sin\alpha)^2 + (R_0-r\cos\theta-r_1cos\alpha)^2
\end{equation}
To find which circle the node crosses first, we also need the angle made by the circle intersections at $B$. Let $\alpha_1$ and $\alpha_2$ be as shown in figures \ref{fig:alpha1} and \ref{fig:alpha2} where $BH$ is a horizontal line. 


\begin{figure}[h]
\centering \vspace{-0.1in}
\includegraphics[width=0.6\textwidth]{images/geo3.pdf}
\vspace{-20pt} \caption[Effect of the proximal Operator]{\small Effect of the Proximal Operator }
\label{fig:alpha1}
\end{figure} 
\newpage
In figure \ref{fig:alpha1}, join $OD$ and drop
a perpendicular from $D$ to meet the extension of the $OB$ at $E$. To avoid clutter, let us 
remove the circles and form figure \ref{fig:alpha11}. 
\begin{figure}[h]
\centering \vspace{-0.1in}
\includegraphics[width=0.6\textwidth]{images/geo4.pdf}
\vspace{-20pt} \caption[Effect of the proximal Operator]{\small Effect of the Proximal Operator }
\label{fig:alpha11}
\end{figure}
Consider $\bigtriangleup DOA$, 
\begin{align*}
	AD^2 &= OD^2 + OA^2 - 2 \cdot OD \cdot OA \cdot \cos(\beta_1+\theta) \\
	R_2^2 &= R_0^2 + R_0^2 - 2R_0^2\cos(\beta_1 + \theta)\\
\end{align*}
\begin{equation}\label{eq:beta1}
	\beta_1 = \cos^{-1}\bigg(1-\frac{R_2^2}{2R_0^2}\bigg) - \theta
\end{equation}
Using $\beta_1$ we can find $\alpha_1$
\begin{align*}
	\tan(\alpha_1 - \theta) &= \frac{DE}{BE} \\
							&= \frac{DE}{OE-OB}\\
							&= \frac{R_0 \sin\beta_1}{R_0\cos\beta_1 - r} \\
	\alpha_1 &= \theta + \tan^{-1}\bigg( \frac{R_0 \sin\beta_1}{R_0\cos\beta_1 - r}\bigg) \\
\end{align*}
\begin{figure}[h]
\centering \vspace{-0.1in}
\includegraphics[width=0.6\textwidth]{images/geo5.pdf}
\vspace{-20pt} \caption[Effect of the proximal Operator]{\small Effect of the Proximal Operator }
\label{fig:alpha2}
\end{figure}
$\alpha_2$ and $\beta_2$ are defined in \ref{fig:alpha2} and \ref{fig:alpha21}. $\beta_2 = \beta_1 + \theta$ by symmetry. Using expression \ref{eq:beta1} for $\beta_1$, we get

\begin{equation}\label{eq:beta2}
	\beta_2 = \cos^{-1}\bigg(1-\frac{R_2^2}{2R_0^2}\bigg)
\end{equation}

\begin{figure}[h]
\centering \vspace{-0.1in}
\includegraphics[width=0.6\textwidth]{images/geo6.pdf}
\vspace{-20pt} \caption[Effect of the proximal Operator]{\small Effect of the Proximal Operator }
\label{fig:alpha21}
\end{figure}

To find $\alpha_2$, consider $\bigtriangleup BGF$ in figure \ref{fig:alpha21}

\begin{align*}
 \tan\alpha_2 &= \frac{FG}{GB} \\
			  &= \frac{FA+AG}{GB} \\
			  &= \frac{R_0 \sin\beta_2 + r\sin\theta}{R_0 \cos\beta_2 - r\cos\theta} \\
\Rightarrow \alpha_2 &= \tan^{-1}\bigg( \frac{R_0 \sin\beta_2 + r\sin\theta}{R_0 \cos\beta_2 - r\cos\theta} \bigg) 
\end{align*}
