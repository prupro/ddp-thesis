
 % This makes the page numbers Roman (i, ii, etc)

%\newpage
%\thispagestyle{empty}
%\mbox{}
% %--------------------------------------------------------------------%
% % APPROVAL SHEET
% %   - for final thesis, you need Approval Sheet. So, uncomment the
% %     \makeapproval command.
% %     it should come after dedication, if dedication is
% %     present. Otherwise it is the first page after title page.
% \makeapproval
% \thispagestyle{empty}
% \begin{center}
%   \begin{Huge}
%     \textsc{\textbf{Certificate}}
%   \end{Huge}
% \end{center}
% % \begin{center}
% % Department of Mechanical Engineering\\
% % Indian Institute of Technology Bombay
% % \end{center}
% 
% \vspace{1in}
% \begin{center}
%  Certified that this Dual Degree Project Stage-I Report titled \\ \textit{``\textbf{Next Generation Distributed Cellular Networks} \\ Architecture and Interference Management''} \\by\\Gaurav Varshney (Roll No. 07D07037) \\ is approved by me for submission. \\To the best of my knowledge, the report represents work carried out by the student.\end{center}
% 
% \vspace{1in}
% \begin{table*}[hb]
% \begin{center}
% \begin{tabular}{lr}
% Date: \hspace{1.9in} & \hspace{1.9in} \textbf{Prof. Abhay Karandikar}\\
% \end{tabular}
% \end{center}
% \end{table*}
% \end{certificate}
\newpage

\thispagestyle{empty}
\begin{center}
  \begin{Huge}
    \textsc{\textbf{Dissertation Approval}}
  \end{Huge}
\end{center}

\vspace{0.2in}
%\begin{center}
% The dissertation entitled \\
%\large ``\textsc{Compressed Sensing for \\Radio Astronomy}''\\ \normalsize by\\ \large \textbf{S Vignesh \\ \vspace{-9pt}\normalsize (Roll No. 100010009)} \\ \normalsize is approved for the degree of\\ \textbf{Bachelor in Technology} in \textit{Electrical Engineering} and\\\textbf{Master of Technology} in \textit{Communications and Signal Processing.\\}\end{center} 

 The dissertation entitled \textit{Compressed Sensing for Radio Astronomy} by \textit{S Vignesh (Roll No. 100010009)} is approved for the degree of \textit{Bachelor in Technology} in \textit{Electrical Engineering} and \textit{Master of Technology} in \textit{Communications and Signal Processing.}


%Bachelor in Technology (Honours) in Electrical Engineering and \\Master of Technology in Communications and Signal Processing.\end{center}
%The dissertation entitled \large ``\textsc{Compressed Sensing for Radio Astronomy}''\normalsize by \large \textbf{S Vignesh  \normalsize (Roll No. 100010009)} \normalsize is approved for the degree of\textbf{Bachelor in Technology} in \textit{Electrical Engineering} and\textbf{Master of Technology} in \textit{Communications and Signal Processing.\\}
\vspace{0.1in}
\begin{flushright}
\textbf{Examiners} \\
\vspace{1.5in}
\textbf{Supervisors}\\
\vspace{1.5in}
 \textbf{Chairman}\\

\end{flushright}
\vspace{0.7in}
Date :    \\
Place :\\
%\begin{table*}[hb]
%\begin{center}
%\begin{tabular}{ll}
%\textbf{Prof.  IITB} & \hspace{0.7in} \textbf{Prof. , IITB}\\
%(Internal Examiner) & \hspace{0.7in} (External Examiner)\\
%\vspace{0.5in}\\
%\textbf{Prof. Rajbabu Velmurugan, IITB} & \hspace{0.7in}  \textbf{Prof. Sibiraj Pillai, IITB} \\ 
%(Guide) & \hspace{0.7in} (Co-Guide)\\
%\vspace{0.5in}\\
%\textbf{Prof. , IITB}\\
%(Chairman)\\
%\end{tabular} \vspace{0.2in}
%\end{center}
%\begin{tabular}{ll}
%Date: & May 22, 2015 \\ \vspace{30pt}
%Place: & Indian Institute of Technology Bombay, Mumbai\\
%\end{tabular}
%
%\end{table*}

%\newpage
%\thispagestyle{empty}
%\mbox{}
%  \newpage
\newpage
\thispagestyle{empty}
\begin{center}
  \begin{Huge}
    \textsc{\textbf{Declaration}}
  \end{Huge}
\end{center}

\vspace{0.5in}

 I declare that this written submission represents my ideas in my own words and wherever others' ideas or words have been included, I have adequately cited and referenced the original sources. I also declare that I have adhered to all principles of academic honesty and integrity and have not misrepresented or fabricated or falsified any idea/ data/ fact/ source in my submission. I understand that any violation of the above will be cause for disciplinary action by IIT Bombay and can also evoke penal action from the sources which have thus not been properly cited or from whom proper permission has not been taken when needed.

\vspace{1.5in}
% \begin{table*}[hb]
% \begin{center}
\hfill \textbf{S Vignesh}\\
\noindent
\begin{tabular}{ll}
Date: & May 22, 2015\\ \vspace{30pt}
Place: & Indian Institute of Technology Bombay, Mumbai\\
\end{tabular}
% 
% \begin{tabular}{lr}
% Date: \hspace{1.9in} & \hspace{1.9in} \textbf{Gaurav Varshney}\\
% \end{tabular}
% \end{center}
% \end{table*}
% 
%\newpage
%\thispagestyle{empty}
%\mbox{}
% 

% \begin{dedication}
  \newpage
  \thispagestyle{empty}    % No page number
%   \setcounter{page}{0}
  % \addtocounter{page}{-1}
%   \chapter*{}            % Required for \vfill to work
%   \thispagestyle{empty}    % No page number

  \begin{center}  \null\vfill
   \textit{\Large To my beloved parents}
  \null\vfill
  \end{center}
% \newpage

% \end{dedication}
%\newpage
%\thispagestyle{empty}
%\mbox{}


\newpage
\thispagestyle{empty}
\begin{center}
  \begin{Huge}
    \textsc{\textbf{Acknowledgements}}
  \end{Huge}
\end{center}
% \begin{center}
% Department of Mechanical Engineering\\
% Indian Institute of Technology Bombay
% \end{center}

\vspace{0.25in}

I would like to express my sincere gratitude to my supervisors Prof. Sibi Raj B Pillai and Prof. Rajbabu Velmurugan for their  invaluable guidance throughout the three years of my association with him. They have always been accessible and willing to clear my doubts and have provided valuable insights that helped me develop new ways to approach the problem and come up with better solutions.\\ %\vspace{0.2in}
% I would like to express my sincere gratitude to my guide Prof. Abhay Karandikar for his invaluable guidance and unwavering support throughout the last four years of my association with him. He has always been accessible and willing to clear all my doubts and inculcate a deeper understanding of the topic by his timely and valuable remarks. He has given critical inputs about my mistakes and given me the opportunity to come up with better and more acceptable solutions. His dynamic attitude and enthusiastic approach to research has instilled in me a motivation to achieve results.\\ %\vspace{0.2in}




\vspace{0.6in}
\noindent May 22, 2015    \hfill \textbf{S Vignesh}\\


%\newpage
%\thispagestyle{empty}
%\mbox{}


\clearpage\pagenumbering{roman} 
\begin{abstract}

We explore the application of compressed sensing for solving problems in radio astronomy where the source images are generally sparse in some domain. We obtain an incomplete set of noisy Fourier measurements of the image through the radio telescope array and the goal is to reconstruct the image by making use of the sparse nature of the images.

In this report we consider the case where we have multiple sets of Fourier measurements corresponding to different images and in addition we have some knowledge about some overlapping information between the  images. By making use of the overlapping information we should be able to perform better reconstruction than in the case where we perform the reconstruction for the images independently. 

We propose a coupled formulation where we solve a joint minimization problem to perform simultaneous recovery of multiple images. We restrict ourselves to the case where we have two images and present an alternating algorithm that solves the joint minimization problem.

We conduct experiments on different classes of images that include images that are sparse in spatial domain, images that are sparse in wavelet domain and images that are sum of a spatial domain sparse component and a wavelet domain sparse component. In all the cases we observed that the coupled formulation that does simulataneous recovery has better performance as compared to when we perform the reconstructions independently.






% Adaptive semi soft handoff technique is used to implement seamless handoff between pico-BSs or macro-BSs. Fast subnet based routing is used in the
% operator's core network. The use of two care of addresses makes movement of a mobile host transparent to the correspondent node.
% Existing packet structures have been modified to implement this scheme. The next stage of the work involves simulation of the above protocols to
% validate their performance.
\end{abstract}
%\newpage
%\thispagestyle{empty}
%\mbox{}
 \newpage

% % 


%--------------------------------------------------------------------%
% CONTENTS, TABLES, FIGURES
% \tableofcontents
% \clearpage{\pagestyle{empty}\cleardoublepage}
% 
% 
% 
% \listoffigures
% \clearpage{\pagestyle{empty}\cleardoublepage}
% %--------------------------------------------------------------------%
% NOMENCLATURE
% \begin{nomenclature}
% \begin{description}
% \item{\makebox[0.75in][l]{$C_1$}} Constant 1
% 
% \item{\makebox[0.75in][l]{$V$}}    Voltage 
% 
% \item{\makebox[0.75in][l]{\$}}     US Dollars
% \end{description}
% \end{nomenclature}

 % Make the page numbers Arabic (1, 2, etc)
% \end{prelude}