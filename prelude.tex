
\newpage

\thispagestyle{empty}
\begin{center}
  \begin{Huge}
    \textsc{\textbf{Dissertation Approval}}
  \end{Huge}
\end{center}

\vspace{0.2in}

 The dissertation entitled \textit{TITLE} by \textit{Prudhvi Porandla (110070039)} is approved for the degree of \textit{Bachelor of Technology} in \textit{Electrical Engineering} and \textit{Master of Technology} in \textit{Communications and Signal Processing.}


\vspace{0.1in}
%\begin{flushright}
%\textbf{Examiners} \\
%\vspace{1.5in}
%\textbf{Supervisors}\\
%\vspace{1.5in}
% \textbf{Chairman}\\

%\end{flushright}
%\vspace{0.7in}
%Date :    \\
%Place :\\
%\begin{table*}[hb]
%\begin{center}
%\begin{tabular}{ll}
%\textbf{Prof.  IITB} & \hspace{0.7in} \textbf{Prof. , IITB}\\
%(Internal Examiner) & \hspace{0.7in} (External Examiner)\\
%\vspace{0.5in}\\
%\textbf{Prof. Rajbabu Velmurugan, IITB} & \hspace{0.7in}  \textbf{Prof. Sibiraj Pillai, IITB} \\ 
%(Guide) & \hspace{0.7in} (Co-Guide)\\
%\vspace{0.5in}\\
%\textbf{Prof. , IITB}\\
%(Chairman)\\
%\end{tabular} \vspace{0.2in}
%\end{center}
%\begin{tabular}{ll}
%Date: & May 22, 2015 \\ \vspace{30pt}
%Place: & Indian Institute of Technology Bombay, Mumbai\\
%\end{tabular}
%
%\end{table*}
\begin{center}
	\begin{tabular}{ccc}
		\rule{60mm}{0pt}        & \rule{10mm}{0pt}       & \rule{60mm}{0pt} \\
		\dotfill                &                        & \dotfill \\
		Examiner                &					     & Examiner \vspace{2cm} \\
		\dotfill                &                        & \dotfill \\
		Supervisor              &                        & Chairman \vspace{2cm} \\
	\end{tabular}    
\end{center}

\vspace{5mm}
\begin{tabular}{lll}
	\rule{40mm}{0pt}        & \rule{50mm}{0pt}       & \rule{60mm}{0pt} \\
	Date: June 20, 2016		&                        & \\
	Place: IIT Bombay       &                        & \\
\end{tabular}
%\newpage
%\thispagestyle{empty}
%\mbox{}
%  \newpage
\newpage
\thispagestyle{empty}
\begin{center}
	\begin{Huge}
		\textsc{\textbf{Declaration}}
	\end{Huge}
\end{center}

\vspace{0.5in}

 I declare that this written submission represents my ideas in my own words and wherever others' ideas or words have been included, I have adequately cited and referenced the original sources. I also declare that I have adhered to all principles of academic honesty and integrity and have not misrepresented or fabricated or falsified any idea/ data/ fact/ source in my submission. I understand that any violation of the above will be cause for disciplinary action by IIT Bombay and can also evoke penal action from the sources which have thus not been properly cited or from whom proper permission has not been taken when needed.

\vspace{1.5in}
% \begin{table*}[hb]
% \begin{center}
	\hfill Prudhvi Porandla\\
	\noindent
	\begin{tabular}{ll}
		Date: & June 20, 2016\\ \vspace{30pt}
		Place: & IIT Bombay\\
	\end{tabular}
	% 

	% \begin{dedication}
		\newpage
		\thispagestyle{empty}    % No page number

	%	\begin{center}  \null\vfill
	%		\textit{\Large To YOU}
	%		\null\vfill
	%	\end{center}
		% \newpage

		% \end{dedication}
		%\newpage
		%\thispagestyle{empty}
		%\mbox{}


	\newpage
	\thispagestyle{empty}
	\begin{center}
		\begin{Huge}
			\textsc{\textbf{Acknowledgements}}
		\end{Huge}
	\end{center}

	\vspace{0.25in}

	I would to express my heartfelt gratitude to my guide Prof. S. N. Merchant for his constant encouragement and patience throughout the project. I would like to thank him for offering moral support to work on the project by placing overwhelming amount of trust in me.    

	I would like to thank Kesav Kaza for his support, advice and presence during every technical discussion and brainstorming session. This work is essentially the summation of all the ideas exchanged during those sessions. I would like to thank all SPANN lab members for making every moment in the lab a learning experience. Special thanks to Zoheb for providing some of the figures. Finally, my wholehearted thanks to the faculty members of Electrical Engineering Department for imparting me with invaluable knowledge. 

	I owe my deepest gratitude to my parents for their relentless support throughout my life and their patience during my DDP work. To the power that drives the Nature without which nothing is possible. 


	\vspace{0.6in}
	\noindent June 20, 2016    \hfill Prudhvi Porandla\\


	%\newpage
	%\thispagestyle{empty}
	%\mbox{}


	\clearpage\pagenumbering{roman} 
	\begin{abstract}

		We explore the application of compressed sensing for solving problems in radio astronomy where the source images are generally sparse in some domain. We obtain an incomplete set of noisy Fourier measurements of the image through the radio telescope array and the goal is to reconstruct the image by making use of the sparse nature of the images.

		In this report we consider the case where we have multiple sets of Fourier measurements corresponding to different images and in addition we have some knowledge about some overlapping information between the  images. By making use of the overlapping information we should be able to perform better reconstruction than in the case where we perform the reconstruction for the images independently. 

		We propose a coupled formulation where we solve a joint minimization problem to perform simultaneous recovery of multiple images. We restrict ourselves to the case where we have two images and present an alternating algorithm that solves the joint minimization problem.

		We conduct experiments on different classes of images that include images that are sparse in spatial domain, images that are sparse in wavelet domain and images that are sum of a spatial domain sparse component and a wavelet domain sparse component. In all the cases we observed that the coupled formulation that does simulataneous recovery has better performance as compared to when we perform the reconstructions independently.
	\end{abstract}
	\newpage
	\mbox{}
	\newpage
